\documentclass{article}
\usepackage{amsmath}
\usepackage{amsfonts}
\usepackage{fullpage}
\usepackage{hyperref}
\begin{document}

\title{Motivation for a $\mathbb{Z}[i]$-sieve}
\date{}
\author{Zeb Engberg}
\maketitle


There are several immediate methods that can be used to find all prime numbers in $\mathbb{Z}[i]$ up to norm $x$. We outline three below.
\begin{enumerate}
\item Use the sieve of Eratosthenes in $\mathbb{Z}$ to generate all prime numbers up to $x$. Hold these rational primes with a sieve array $A$. Note that a Gaussian integer $a + bi$ with $a > 0$ and $b\ge 0$ is prime in $\mathbb{Z}[i]$ if and only if $\operatorname N(a + bi)$ is prime in $\mathbb{Z}$ or $b = 0$ and $a$ is a prime in $\mathbb{Z}$. This sieving process takes $O(x \log \log x)$ steps whereas checking each Gaussian integer $a + bi$ with $\operatorname N(a + bi) \le x$ for primality through the sieve array $A$ takes $O(x)$ steps. Hence the entire runtime of this algorithm is $O(x \log \log x)$.

\item Start as in the previous algorithm by generating all rational primes $p\le x$. If $p \equiv 3 \mod 4$, then $p$ is also prime in $\mathbb{Z}[i]$. If $p \equiv 1 \mod 4$, then $p$ factorizes in $\mathbb{Z}[i]$. Finally, if $p = 2$ we have the factorization $2 = - i (1 + i) ^ 2$.

To factor those $p \equiv 1 \mod 4$, first compute a primitive fourth root $r$ of 1 in $\mathbb{Z}/p\mathbb{Z}$. Such an $r$ satisfies $r ^ 2 + 1 \equiv 0 \mod p$. Using a random algorithm, we can expect to find such an $r$ in $O(1)$ time per prime. Once $r$ is found, calculating the greatest common divisor of $p$ and $r + i$ in $\mathbb{Z}[i]$ will yield a factor $a + bi$ of $p$. Calculating this $\gcd$ will take at most $O(\log p)$ steps. Summing this over all primes $p\equiv 1 \mod 4$ with $p\le x$, the number of sieving steps is still the dominate term in the complexity estimate, and this algorithm requires $O(x \log \log x)$ steps just as in the previous algorithm.

\item Rather than starting with a sieve in the rational integers $\mathbb{Z}$ and using information about the splitting behavior of primes $p\in\mathbb{Z}$ lifted to $\mathbb{Z}[i]$, it is possible to sieve directly in $\mathbb{Z}[i]$. The argument giving the runtime of the sieve of Eratosthenes in $\mathbb{Z}$ also shows that this $\mathbb{Z}[i]$-based sieve requires $O(x \log \log x)$ steps.
\end{enumerate}

 
In this project, we implement the third algorithm described above. Of the three algorithms, the third is the most difficult to design because it requires stepping through a 2-dimensional array rather than the usual 1-dimensional array in the standard sieve of Eratosthenes. If all three algorithms have the same runtime, (and because the \href{https://github.com/kimwalisch/primesieve}{primesieve project} renders the second algorithm the fastest in practice), why would sieving directly in $\mathbb{Z}[i]$ be of any use?

Sieving directly in the Gaussian integers via the third algorithm becomes useful when working within \emph{segments} of the complex plane. Consider the following example: Given $x, y, z$, and $w$, suppose we wish to find all Gaussian primes $a + bi$ with $a \in [x, x + z)$ and $b \in  [y, y + w)$. Here we are thinking of $x$ and $y$ as large and $z$ and $w$ as small relative to $x$ and $y$. Sieving directly in $\mathbb{Z}[i]$ to find such primes would require
$$O(B \log \log B) + O\left(\sum_{ N(\pi) \le B } \frac{zw}{\operatorname N(\pi)}\right)$$
steps, where $B = ((x + z)^2 + (y + w)^2)^{1/2}$. The first term above corresponds to finding all primes $\pi \in \mathbb{Z}[i]$ with norm up to the square-root of the largest element in the sieve array $B$, and the second term corresponds to crossing off multiples of these primes $\pi$ in the sieve array indexed by $[x, x + z) \times [y, y + w)$. If $x$ is larger than $y, z, w$ (say), this big-$O$ estimate simplifies to
\begin{equation}
O((x + zw) \log \log x).
\end{equation}

An analog of one of the first two $\mathbb{Z}$-based sieving algorithms from above would require the computation of rational primes $p \equiv 1 \mod 4$ in the interval
$$\left[ x^2 + y^2, (x + z)^2 + (y + w)^2 \right).$$
Once the dust settles, such an algorithm would require
\begin{equation}
O(xz\log\log x)
\end{equation}
steps. If $x$ is large and $z$ is moderate in size, the estimate in line (2) is significantly larger than the estimate in line (1) obtained from sieving directly in $\mathbb{Z}[i]$.

In addition to employing segmented sieving in rectilinear boxes $[x, x + z) \times [y, y + w)$, we can also use segmented sieving to generate Gaussian integer primes in sectors in the complex plane. Given angle measures $0 \le \alpha < \beta < \pi/2$, consider the primes $\pi \in \mathbb{Z}[i]$ with $\operatorname N (\pi) \le x$ and $\alpha \le \arg \pi \le \beta$. Rather than generating all primes $\pi$ with $\operatorname N (\pi) \le x$ then filtering by angle, we can instead sieve directly in the sector defined by angle measure in the interval $[\alpha, \beta]$. As $\beta - \alpha$ becomes small, $\mathbb{Z}[i]$-sieving will become increasingly more efficient than sieving in $\mathbb{Z}$ .

To summarize:
\begin{itemize}
\item If one wishes to generate all Gaussian integer primes up to some norm bound, use a $\mathbb{Z}$-based sieve and split the primes $p\equiv 1 \bmod 4$ into Gaussian integer primes.
\item If one wishes to generate Gaussian integer primes in some smaller subset of the complex plane (a box, a sector), direct sieving in $\mathbb{Z}[i]$ becomes more efficient.
\end{itemize}
\end{document}